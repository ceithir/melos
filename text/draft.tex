\documentclass{report}

\usepackage[T1]{fontenc}
\usepackage[utf8]{inputenc}
\usepackage[greek,english]{babel}

\usepackage[hidelinks]{hyperref}
\hypersetup{
    colorlinks,
    linkcolor=blue,
}

\usepackage{comment}

\newcommand{\dcomment}[1]{
	\emph{#1}
	\\
}

\newcommand{\speaker}[1]{
	\textbf{#1}: 
}

\newcommand{\glink}[2]{
	\hyperref[#1]{#2}
}

\newcommand{\gsection}[1]{
	\section*{#1}
	\label{#1}
}

\title{LaTeX draft for \\"Ancient Greece Yuri Visual Novel"}
\author{Skarn}

\begin{document}

\selectlanguage{english}

\maketitle

\chapter*{Where it all starts}

\dcomment{
	Night sky background
}

I don't know how and when it started.\\

\dcomment{
	Same background, one double paragraph of poetry filling most of the screen instead of scripts, with both the original Greek and its English translation, and the narration in the standard textbox
}

\begin{otherlanguage}{greek}
Ποικιλόθρον᾽ ἀθάνατ᾽ ᾽Αφρόδιτα,\\
παῖ Δίος, δολόπλοκε, λίσσομαί σε\\
μή μ᾽ ἄσαισι μήτ᾽ ὀνίαισι δάμνα,\\
πότνια, θῦμον.
\end{otherlanguage}

On the throne of many hues, Immortal Aphrodite,\\
child of Zeus, weaving wiles--I beg you\\
not to subdue my spirit, Queen,\\
with pain or sorrow
\\

I remember that morning on the road.

I had finally managed to have her talk a little about her past. For a second, for a few words, she opened the door of her heart to me, and I felt like I had unlocked the path to Elysium.\\

\begin{otherlanguage}{greek}
ἀλλὰ τυίδ᾽ ἔλθ᾽, αἴποτα κἀτέρωτα\\
τᾶς ἔμας αὔδως ἀίοισα πήλοι\\
ἔκλυες, πάτρος δὲ δόμον λίποισα\\
χρύσιον ἦλθες
\end{otherlanguage}

but come--if ever before\\
having heard my voice from far away\\
you listened, and leaving your father's\\
golden home you came
\\

I remember that afternoon in the pond.

We bathed together, not for the first time nor the last. But from that moment onwards, I was suddenly well aware of our common nudity and physical closeness.\\

\begin{otherlanguage}{greek}
ἄρμ᾽ ὐποσδεύξαια· κάλοι δέ σ᾽ ἆγον\\
ὤκεες στροῦθοι περὶ γᾶς μελαίνας\\
πύκνα δινεῦντες πτέρ᾽ ἀπ᾽ ὠράν᾽ αἴθε-\\
ρος διὰ μέσσω,
\end{otherlanguage}

in your chariot yoked with swift, lovely\\
sparrows bringing you over the dark earth\\
thick-feathered wings swirling down\\
from the sky through mid-air
\\

I remember that evening in a tavern overflowing with people.

Our table was crowd, but that did not bother her. Actually, she was smiling, joking, \emph{flirting} with the man next seat. I kept my best polite face. Inside my chest the infernal fire of jealousy and the chilling wind of despair were battling each other.\\

\begin{otherlanguage}{greek}
αἶψα δ᾽ ἐξίκοντο· σὺ δ᾽, ὦ μάκαιρα\\
μειδιάσαισ᾽ ἀθάνατῳ προσώπῳ,\\
ἤρε᾽ ὄττι δηὖτε πέπονθα κὤττι\\
δηὖτε κάλημι
\end{otherlanguage}

arriving quickly--you, Blessed One,\\
with a smile on your unaging face\\
asking again what have I suffered\\
and why am I calling again
\\

I remember that night in a particularly desolate region.

We had set up camp in a small grove. It was summer, it was hot, so we decide to sleep under the stars. She indeed did sleep. As for myself, I spent the entire night contemplating her, watching the slow up-and-down of her bust as she quietly breathes.\\

\begin{otherlanguage}{greek}
κὤττι μοι μάλιστα θέλω γένεσθαι\\
μαινόλᾳ θύμῳ· "τίνα δηὖτε †πείθω\\
ἄψ σ᾽ ἄγην† ἐς σὰν φιλότατα; τίς τ᾽, ὦ\\
Ψάπφ᾽, ἀδίκηει;
\end{otherlanguage}

and in my wild heart what did I most wish\\
to happen to me: "Again whom must I persuade\\
back into the harness of your love?\\
Sappho, who wrongs you?
\\

I remember our first encounter.

I knew nothing of her except of her skills with the spear and the snare. A shady character, refusing to reveal anything personal about herself. A probable fugitive on the run from the authorities, a potential thief, bandit, murderer.

Yet I still entrusted my safety to her. Out of necessity I tried to convince myself then. But perhaps, already, unconsciously, under the whispers of a part of myself older and far more powerful than my reasonable mind.\\

\begin{otherlanguage}{greek}
καὶ γάρ αἰ φεύγει, ταχέως διώξει,\\
αἰ δὲ δῶρα μὴ δέκετ᾽, ἀλλὰ δώσει,\\
αἰ δὲ μὴ φίλει, ταχέως φιλήσει,\\
κωὐκ ἐθέλοισα".
\end{otherlanguage}

For if she flees, soon she'll pursue,\\
she doesn't accept gifts, but she'll give,\\
if not now loving, soon she'll love\\
even against her will."
\\

I remember our last discussion.

We were waiting outside the stadium, before dawn, before this long day of festivities starts. More daring than usual, I had not so subtly directed the conversation on the various interpretations of my repertoire's works, and was watching her closely, attentive to any sign, any reaction which could have given me hope.

I saw none.\\

\begin{otherlanguage}{greek}
ἔλθε μοι καὶ νῦν, χαλέπαν δὲ λῦσον\\
ἐκ μερίμναν, ὄσσα δέ μοι τέλεσσαι\\
θῦμος ἰμέρρει, τέλεσον· σὐ δ᾽ αὔτα\\
σύμμαχος ἔσσο.
\end{otherlanguage}

Come to me now again, release me from\\
this pain, everything my spirit longs\\
to have fulfilled, fulfill, and you\\
be my ally
\\

I remember these moments, and many others, small and big, casual and special, every precious little brick of the time I spent with her.\\

\dcomment{
	Poem box disappears, leaving only the background and the standard textbox
}

A final strum, and the last note escapes the lyre, rises into the air, rings through the night. In its mourning, it echoes the end of the song, and is answered with applauses and cheers.\\

\dcomment{
	Background finally fade in
}

About twenty or thirty persons noisily show their appreciation of my recital of Sappho's classic. I have had bigger audiences, but at that time of the night, with everyone spread out in the many rooms and gardens of the building, that's not so bad.\\

I smile and salute, then climb down the rostrum as another artist ascends. In a couple of seconds, I cease to be the center of attention, fall back to the condition of simple spectator of someone else's show. I sit quietly on the side of a pedestal, half hidden in the shadow of the statue it bears.\\

And I almost break into a million pieces.\\

The \emph{Hymn to Aphrodite} is a powerful piece, one people expect and enjoy, one I have played many times in the past, and that I will probably play many times in the future too.

But for several months now, I have been unable to hum its melody without picturing a specific face, without imagining a certain pair of beautiful eyes looking at me while I'm singing, without dreaming about one unique person listening to the words between the lines and understanding the special meaning this song has for me.\\

And this is but the most harmless of my recent quirks. I have turned obsessive, possessive. I sleep little and badly, cannot bear the shortest of separations, constantly observing her and overinterpreting each and every of her words and actions.

I do my best to hide it, and as a professional artist trained to keep face and smile in all circumstances, I'm pretty sure I have managed to keep the mask of sanity and happiness on until now.\\

But deep inside, I'm crumbling, I'm on the verge of breaking, at the gates of madness. The flames are in me, they want to go out, but I keep them locked inside, and so they consume my soul.\\

More than the passion itself, it's the silence which is killing me. I haven't told anyone. Not her. Not anyone else, to be sure this would not reach her ears somehow. In fear of the consequences of such reveal.\\

I have observed her for days and nights. Yet, I never found a single hint that she may feel about me the same way that I feel about her.

But I have grasped enough to understand she was born in a culture of power and steel which does not value highly romance, a world where reproduction takes precedence over proper love. And that she still lives by the spirit of her education if not by its customs.\\

And I'm afraid. Very afraid. Of what will happen when she will discover my passion, my desires.\\

Afraid of bewilderment.

Afraid of rejection.

Afraid of fear.

Afraid of anger.\\

Afraid this will break our special relationship.\\

Afraid that by wanting too much of her, I will lost her whole.\\

\dcomment{
	Music change? Simply a lyre (harp) tuning sound effect as transition?
}

As a background noise for my worries, the man on the scene started playing. I recognize a composition from Alcaeus. A supposedly appropriate answer to my own choice, as he was Sappho's senior and rival. But a poor idea in the end, as the stanzas I hear simply cannot compare to the majesty of Sappho's texts. She was the uncontested Tenth Muse, and the artistic world still mourns her death a couple of decades ago.\\

The nagging melody nonetheless calms me down a little. I review my options. This afternoon, I took part in one of the main events of the festival, in the city theater, under the gazes of hundreds of people. A grand contest of poetry in whom many artists tried their luck, including the one distracting us right now.\\

Triumphing in such a competition is usually a life's goal for a singer, but this was not my main concern this year. However, I did plan my day with an extreme precision. As soon as my early and obligatory performance ended, I retreated back to the shadows, escaped discreetly the theater to hide myself in the nearby woods.\\

Then I sat, on the ground of a sunny clearing. Alone, while the whole city was directly or indirectly focused on the festival, from the shouting people on the theater's tiers to the humblest food vendor. To think. Think far away from the constant noise and movement of the ordinary life. Think without the weight of exhaustion and the terrors of the night overtaking me.\\

And after a few hours, when I knew I couldn't hide or flee anymore, I made my decision. It's this night or never. The moment couldn't be more perfect. Today, the city is – we will are – giving thanks to Aphrodite. Today, we both shone in our own specialty, prove ourselves to the world, reminded it we exist by our own rights. This night, right now, her mood should be at its best, on the edge of ecstasy.\\

If I don't talk to her this night, I know I will never have enough will to do it again. That my mouth will stay shut and my heart closed.\\

So... I just need to stand up, walk away from this quiet corner of the property where a few artists and enthusiasts gathered for some impromptu rehearsals, find her in the crowd, take her apart, and open my heart to her.\\

As simple as that.

\begin{itemize}
	\item \glink{Wait 1}{Wait}
	\item \glink{Go 1}{Go}
\end{itemize}

\gsection{Go 1}

I raise my head. I look at the nearby exit. So close. I just have to get up. Walk a few steps. Simple. Easy.

\begin{itemize}
	\item \glink{Wait 1}{Wait}
	\item \glink{Go 2}{Go}
\end{itemize}

\gsection{Go 2}

As I'm about to stand up, the artist on the scene rotates once more, and I hear the first note of a song I really like. Couldn't I just stay there at least until it ends?

\begin{itemize}
	\item \glink{Wait 1}{Listen}
	\item \glink{Go 3}{Go!}
\end{itemize}

\gsection{Go 3}

I try to stand up. My knees are shaking like a set of knucklebones. I'm shivering like an old sick lady. There is a jar of wine and a cup not far. Perhaps a drop could help.

\begin{itemize}
	\item \glink{Wait 1}{Have a drink}
	\item \glink{Go 3}{Just go!}
\end{itemize}

\gsection{Go 4}

I lash myself mentally. I'm still trying to push back the ineluctable. To flee what is hard and fearsome. But I cannot.\\

I gather my will.

I clench my teeth.

I rise.\\

I'm not stable on my feet but I nonetheless stumble on the short distance separating me from the main building. As soon as I've passed the separation door, that I'm plunged once again into the living heart of the party, noisy, hot, full of people, my faculties return to me. I'm no less afraid.

But once you jumped into the water, you have no more choice than to swim.\\

\glink{Searching for her}{Continue}\\

\chapter*{A talk}

\gsection{Searching for her}

I advance quickly through the many rooms of the building, getting around and stepping over the many revelers crossing my path with a consummate skill. I have worked at many similar parties before, and experience taught me how to move with ease despite the chaotic crowd.\\

I make a breach through human walls with a smile and a gentle but firm arm, adroitly avoid stepping on discarded leftovers and other unsavory things, nicely dismiss people asking for a song.\\

While my reflexes alone handle my movements and keep my pace, I gaze at every single person, my sight jumping from face to face, hunting the only prey worth it. A hundred different eyes meet mine, but not the ones I desire. Until they finally lock into perfection.\\

\dcomment{
	Close-up on the eyes? "Special attack" style, with a narrow rectangle in the middle of the screen
}

They may right now display the slight luster symptomatic of an important consumption of alcohol. But they still shine like twin suns. The reflected flames of the candles dance upon the dark and brown ocean, shifting with the emotions of their owner. Enlarged from the surprise of the sudden encounter. Contracted as she observes the intruder. Sparkling when she recognizes me.\\

She half rises, hemmed in her movements by the food plate on her knees, but I stop her with a sign of the hand and sit at her side at the extremity of the narrow bench, pushing her a little. She holds back a grin and I show a mischievous smile. Her military mind had naturally chosen a position from which she could have easily fled. An advantage I just negated, cutting her way out.\\

\dcomment{
	First appearance of Antiope sprite (little by little?)
}

From so close, it's obvious I'm at least half a head taller than her, which never ends to surprise me. She looks big in everyday life, thanks to her headstrong confidence and athletic build, but is actually pretty average when using factual measurements.\\

My shoulder brushes against her own, my robe touching the skin left nude on that place by her old style of clothing. Some judge it barbaric. She judges it practical. I judge it sexy as...\\

\speaker{Her} "Ambrosia, where have you been? I looked after you when the party started, but you were nowhere to be found."\\

Her voice returns me to the real world. Sort of. Before answering, I savor my own name in her mouth like it is the divine nectar it refers to. She pronounces it with a slight but unmissable accent, giving it a strange sonority unique to her, a cute quirk I have learned to enjoy.\\

On the opposite side, despite my talents as a singer, I'm unable to breath proper life into her own name. I pronounce it perfectly, but it sounds so bland when it should ring like the thunder in the summer sky. \emph{Antiope!} A warrior name, an amazon name, a mythological name, yet one escaping me as her bearer escapes me.\\

I display a pale smile. I could easily answer truthfully. I could as easily direct the conversation in another direction. Or even more easily remain silent.

\begin{itemize}
	\item \glink{Approaching the rift}{Truth}
	\item \glink{Sport}{Distraction}
	\item \glink{Silence}{Silence}
\end{itemize}

\gsection{Approaching the rift}

\speaker{Ambrosia} "I needed a break. To put some distance between me and the crowd. To think."\\

I barely recognize my own voice. It sounds so... depressed. Barely a whisper, filled with shrills and holes, all music gone from it. She blinks, her face suddenly serious, concerned.\\

\speaker{Antiope} "Ambrosia, are you alright? You're not acting like yourself. Has someone done something to you?"\\

Yes dummy, someone has. That someone is you.

\begin{itemize}
	\item \glink{It's nothing}{Reassure her}
	\item \glink{The garden}{Ask if we can discuss it somewhere else}
\end{itemize}

\gsection{Silence}

I remain silent, dismissing the question with a mysterious smile and a wave of the hand. She does not insist, and hands me a bunch of pastries wrapped in a small piece of fabric. I accept one and starts eating slowly, with little mouthfuls, not in a hurry to speak. She devoured one with far less decorum and I remark that she has piled up enough food to feed a whole army. Following my glance, she displays a all teeth out smile.\\

\speaker{Antiope} "Eat like a wolf. As much as you can while you can, so to be ready for when you will have nothing eat."\\

Typical. She knows a seemingly endless list of rules, proverbs, little tales implying carnivorous animals and survival. This one is actually pretty tame compared to the most bloody, dark and depressing you've heard from her.\\

And she does live by them. Always stuffing herself to the limit each time the food is free. It's a wonder how she keeps her physique, even if her stupidly hard and harsh daily routine probably helps a lot.\\

Yet another habit one could find ridiculous at first, but touching once you realized it has been forcefully implemented into her, a scar from a childhood she refuses to talk about.\\

She engulfs plate after plate, swallowing exactly one full cup of wine between each of them with a fascinating regularity. Even if the alcohol is heavily watered, she clearly already took more than her share, and I worry she will soon fall in an advanced state of inebriation.

\begin{itemize}
	\item \glink{Stupor}{Let her do that she wants}
	\item \glink{Sobriety}{Stop her}
	\item \glink{Drunk}{Join her}
\end{itemize}


\gsection{The garden}

\speaker{Ambrosia} "Can we talk about this in a calmer and quieter place?"\\

This statement does not reassure her a bit. She looks around, appears to notice the many people around us, even if most actually pay no attention to our little show, busy that they are with their friends or drinks, but still contributing to the stifling atmosphere of the overfull room.\\

She then reacts with her natural straightforwardness. Grabbing me by the wrist, she walks right through the crowd, taking the shortest route to the nearest exit. If I got here like a breeze, slipping in cracks and leaving no trace of my passing behind me, Antiope is a ram, a spur, elbowing her way out, breaking any resistance like a charging bull.\\

I should probably scold her for her rude behavior. But I cannot resolve myself to do it. Partly because I realize I really want to go out. Partly because I'm somehow happy, even if this happiness is tainted with guilt, that she is worried about me, that did not hesitate for one second to discard all decorum for me.\\

\dcomment{
	Reuse the garden description, because I'm that lazy
}

Now what we're outside, that the cold of the night contrast with the fire of our bodies, we take a break then a walk, strolling lazily through the gardens.\\

\speaker{Antiope} "Feeling better?"\\

I turn my head, my heart pounding, at the first actual words, and not grunts, coming out of Antiope's mouth since she took charge. And I nod to them like an afraid mouse.\\

\speaker{Antiope} "So, what's all that about?"\\

I rise my hand to ask for a pause. The  building is touching the close forest, and we certainly did get over the invisible frontier between the gardens and the woods. The trees are taller, the sky less visible, only a few stars and the moon peeking through the branches. We bath in the false silence of nature, these many little noises of an intense vegetable and animal life void of humans. Only the two of us alone in the world.\\

\begin{itemize}
	\item \glink{Blunt}{Straight to the point}
	\item \glink{Backstory}{A subtle approach}
\end{itemize}

\gsection{Stupor}

\dcomment{
	Not uninteresting scene, but need to be delayed later... Or at least the transition is too brutal. Insist on the time passing and Antiope letting herself loose
}

With each new plate, with each new cup, her movements slow down, her actions are more hesitant, more imprecise. Until finally, her febrile hands let go a ripped tomato, which splashes into the ground.\\

She looks at the stain with absent eyes for several couples of seconds, and suddenly she rises from her seat and starts stumbling in the approximate direction of the nearest exit.\\

Seeing her in pain, I immediately stand up and help her. My left arm around her shoulders, I guide and support her all the way out. As we laboriously walk together, trying not to bump into anybody, I can feel her erratic, hot, wine-laden breath against my neck, her moist skin under my palm. She really is in a poor state.\\

After an eternity, we finally meet the fresh air of the outside. A beautiful garden shows itself before us. It was full of people earlier, but many hours have passed since, the night is now at its coldest, and the place has been almost deserted.\\

I drag Antiope away to a bench near a fountain. The stone is slightly humid, extremely cold, and I worry that letting her lie on it could be an incredibly stupid idea. But she just sits – collapses – there.\\

She remains like this, motionless, expressionless, her eyes staring at her feet, for at least a good minute. Then she finally lets go a weak groan, straightens up her head to look at me and speaks with difficulty.\\

\speaker{Antiope} "I may have overdone it."\\

I feel fire rising inside my throat. Anger.\\

\speaker{Ambrosia} "Yes. Yes you did."\\

\speaker{Antiope} "I didn't realize I had already drunk so much. We celebrated a first time after this morning games and..."\\

\speaker{Ambrosia} "Antiope! That's not the problem."\\

I almost yelled the last sentence. I am furious, burning with a sore rage.\\

\speaker{Ambrosia} "That's not the first time that happened! You eat and you drink as if trying to killing yourself from asphyxia and drowning! I don't care about your proverbs, this is just sick."\\

She stares at me with empty eyes. Stammers a few syllables. Stands up. Go to the pond. Splashes her face with cold water. Returns to the bench. Sits up again, looking a little better this time. And answers.\\

\speaker{Antiope} "You know of this Spartan story with the fox?"\\

\speaker{Ambrosia} "The one where a little boy would rather have his insides rip apart by the fox he stole than to be caught red-handed? Yes, I know of it."\\

\speaker{Antiope} "Yes, that one. What do you think of it?"\\

I hesitate, disoriented with the sudden change of subject.\\

\speaker{Ambrosia} "A gory tale tailored for a quick spread and a long-lasting legacy. Bloody, violent, shocking, everything a good urban legend needs to endure. It can also be interpreted as much as a tribute to the courage of Spartans as a denunciation of the cruelty of their education methods, and thus is told by both admirers and the enemies of this city."\\

\speaker{Antiope} "Do you think it's based on a true story?"\\

\speaker{Ambrosia} "I... don't know. Probably not, even if it appears to take its roots in some real Spartan customs."\\

\speaker{Antiope} "Yes, teaching how to steal to your future soldiers is actually a pretty rounded idea."\\

Antiope is speaking quickly, too quickly, each word mashing with the next, her accent more pronounced than ever.\\

\speaker{Antiope} "A marching army is made of thousands of people, all needing to eat. And most often than not, for the core of the troops, after a few days, all their rations are gone and they have to live off the land. A ridiculous euphemism for stripping bare any field, any tree, any granary, any house. For a nobody in such cohorts, the art of soulless plundering is perhaps more important than knowing how to handle your weapon. The main fight may never come for many political reasons, but your stomach still aches every day."\\

\speaker{Antiope} "And, sometimes, an expedition can turn \emph{really} bad. The war may have trailed for so long that literally every edible thing around has already be ransacked. Far more people have already died from the many diseases corrupted water and rotten food bring than on the actual battlefield.\\

\speaker{Antiope} "Then the suffering army can be crushed utterly regardless of its many sacrifices, pushed into a retreat so chaotic that many small isolated groups are left behind, and must now make their way back on their own, in hunger and fear. For a perfect agony, you can even add some rain or snow."\\

\speaker{Antiope} "A lot more will die. But quite a number will also somehow survive. Among them, some will have sacrificed all morality, decency, humanity to escape the sirens of Inferno. They succeeded in returning to their home, but they will never be the same, their behavior forever changed."\\

Antiope suddenly stops talking and shuts her mouth tight. She looks frightening, her eyes burning with a dark hate I have never seen before.\\

\begin{itemize}
	\item \glink{Posttraumatic}{Comfort her}
	\item \glink{Fleeing the past}{Keep silent}
\end{itemize}

\gsection{Posttraumatic}

I softly put my hand on her shoulder.\\

\speaker{Ambrosia} "It's okay."\\

Antiope's fingers lock on my wrist with tremendous strength and she pushes back my arm like she was deflecting some sword blow, getting to her feet in the same movement. My heart starts beating furiously again, but from fear this time. I see death in her eyes, an unsheathed killing instinct.\\

\speaker{Ambrosia} "Antiope, please..."\\

My voice is but a whisper, but seems to have an effect nonetheless. Some emotions return to her sight, and she releases me. I move back a few steps, my hurt wrist pressed against my chest, the skin having turned red where she pressed it.\\

\speaker{Antiope} "I'm sorry..."\\

Her voice is pitiful, far from her ordinary confidence, her face has crumbled into a mask of distress.\\

\speaker{Antiope} "I didn't want to..."\\

\speaker{Ambrosia} "Are you ready to speak without trying to break my arm?"\\

My own voice is cold as ice. My wrist hurts, but my heart hurts far more. I feel betrayed.\\

Antiope freezes, looks at me attentively. I can only imagine what she is seeing, but when she answers, it's with a single, weak, almost inaudible, syllable.\\

\speaker{Antiope} "Yes."\\

We walk to another part of the garden. It's no less cold, just a little less humid, but I couldn't stand to remain near that damn pond for one more minute. We find another bench, and sit in such a way we're facing each other, me with my legs on the stone and my arms around my knees, Antiope in an awkward three-quarters position, her legs balancing and her torso turned in my direction.\\

\speaker{Antiope} "I'm sorry."\\

\speaker{Ambrosia} "For?"\\

\speaker{Antiope} "For my behavior. For having attacked you in all possible manners when you were just trying to prevent me from hurting myself. For being an idiot in general."\\

\speaker{Ambrosia} "And more precisely?"\\

\speaker{Antiope} "It's hard to put into words... I sometimes feel like the scorpion desperately resisting the impulse to sting. I'm a living pile of stupid, dangerous reflexes and biases. Do you know that the first nights we passed together, I barely got any sleep because I didn't completely roll out you would rob me as soon as I close my eyes? Despite everything and anything proving me you were no bandit?"\\

You try to appear unfazed, but this piece of information is indeed new to you.\\

\speaker{Antiope} "I'm fighting it every day that... \emph{paranoia}. I don't know if it's the proper word. Sort of fear of everything. But when I'm stressed, exhausted, drunk, my impulses get the best of me."\\

\speaker{Antiope} "Usually, I manage to keep them in check regardless, or at least to hide myself when they're at their worst. But today has been a long day. I have been tired since midday and partially drunk since early afternoon."\\

She pauses.\\

\speaker{Antiope} "I understand this is no excuse, only explanation. I know my weaknesses first and foremost. Some of these troubles may have been implanted into me by circumstances and bad teaching but it's still my responsibility that they do not hurt others. Especially those I care about."\\

I sigh. Lengthily.\\

\speaker{Ambrosia} "Why didn't we have this conversation earlier?"\\

\speaker{Antiope} "I don't really know. The last times I went overboard... Sometimes, I manage to escape these talks... And the other times..."\\

\speaker{Ambrosia} "Yes. I know. I'm the one who fled."\\

I look at her right in the eyes, seeing my own face reflected in her pupils. Yes, there are a few times when I just kept silent on Antiope sick habits. Not only about alcohol, but also concerning her violent behavior. Today the first time she ever hurt me, but there were several mornings when her face or fists were bearing the marks of fights she got herself into. And I never dared to ask.\\

We're a nice couple of cowards and idiots.\\

\begin{itemize}
	\item \glink{Declaration}{Now}
	\item \glink{Friends}{Never}
\end{itemize}

\gsection{Fleeing the past}

Time passes.

Slowly.

Antiope does not budge a muscle.

I don't dare moving either.\\

We stay like this for minutes? Hours? I can't say. I just know that after an eternity in the cold, Antiope finally stands up and whispers:\\

\speaker{Antiope} "Thanks."\\

And on these words, she walks away into the night.\\

\glink{Masks}{Continue}

\gsection{Blunt}

I feel like if I don't say it \emph{now}, I will never say it. That if I don't tell things clearly and loudly, they're going to be misinterpreted.\\

So I turn myself to face Antiope. I breath hardly, too quickly, hyperventilating like mad. I force my elocution to form proper, articulate words, using all my professional tricks to sound as clear as humanly possible.\\

\speaker{Ambrosia} "I. Love. You."\\

As soon as the last syllable dies, I start panicking again. My whole body is paralyzed with fear, I'm unable to move, to look away, to speak. But under the skin, every little part of my organism is racing furiously.\\

Antiope is bewildered. Clueless. My sudden outburst takes several seconds to reach her brain, and a lot more to be processed. She blinks a few times, then repeats :\\

\speaker{Antiope} "You... Love... Me?"\\

I nod, a large smile engraved on my face. Antiope appears embarrassed, her right hand fidgeting on her left hip. She struggles to find her words, her tone pitching irregularly.\\

\speaker{Antiope} "That's an \emph{unexpected} declaration. Were you \emph{stressed} because you were \emph{planning} it?"\\

I nod again, even if I refrain a frown at the way she presents it. Like if the whole day had gone according to my design to trap her in this distant place. And she acts like a solider suspecting an ambush. Actually, her tic is similar to playing with the pommel of an invisible sword.\\

\speaker{Antiope} "I'm sorry but I have no \emph{romantic} interest in you Ambrosia."\\

She is really defensive, talking  like... Like how she used to talk when we first met. Keeping her distance, speaking little. She did defrost with time, but she suddenly is as icy as before.\\

\speaker{Antiope} "Is that all you wanted to tell me?"\\

I nod again, unable to do anything else.\\

\speaker{Antiope} "Fine. I think we should go back then. It's cold there."\\

I agree silently and follow her as she traces her way back to the building without looking behind even once.\\

\glink{Masks}{Continue}

\gsection{Backstory}

I don't know how to approach the subject. I have devised a thousand ways to do so, but none seems right. Antiope can have extreme instinctive reactions, and I don't want any bad word to trigger one. I decide to start by the beginning.\\

\speaker{Ambrosia} "Do you remember our first days together? We were going to Delphi by the land road, a trip of a few weeks through scarcely populated regions."\\

Antiope cocks an eyebrow but answers.\\

\speaker{Antiope} "Of course I remember. A damn eventful trip. We even nearly got attacked by bandits once. If we hadn't spot their lookout and made a detour, things would have turned ugly."\\

She displays a crooked smile at the evocation of missed violence.\\

\speaker{Ambrosia} "Do you remember what we discussed during these long days and nights?"\\

Antiope thinks for a few seconds, gathering her memories.\\

\speaker{Antiope} "Actually... I don't. Is there something I've forgotten? Something you told me... or that I told you?"\\

She suddenly raises up her defences. The idea of having let go any dangerous piece of information about herself in an instant of weakness triggered her alarm signal hard. I reassure her, even if slightly unnerved.\\

\speaker{Ambrosia} "Far from it. Actually, it's normal you don't remember... Because there is nothing worth remembering. We pretty much didn't talk to each other back then. Sure, we did exchange civilities and factual data about our environment and planning... But that's all. No actual conversation."\\

\speaker{Antiope} "Ah, yes... I wasn't really talkative back then."\\

\speaker{Ambriosa} "\emph{Withdrawn} would be a more exact term. But slowly, you began to speak more openly. Can you explain why?"\\

\speaker{Antiope} "I never really thought about this. I guess it's happened naturally. At first, we were pretty much strangers traveling together, and I was more wary than friendly. I'm not the kind to fake amicability in such situations. It's only with time passing that things got better."\\

\speaker{Ambriosa} "How?"\\

\speaker{Antiope} "How?"\\

She seems completely lost.\\

\speaker{Ambrosia} "How did they get better?"\\

\speaker{Antiope} "I guess... I leaned to trust you, began to relax more when being with you and... Damn, Ambrosia, you're far better are putting this feelings' stuff into words than me. Why the questioning?"\\

\speaker{Ambrosia} "Antiope, believe me, it's important."\\

\speaker{Antiope} "You're trying to make me say something, even I can tell. But what? That we used to be strangers under the same roof before we actually turn to friends? That's not an abnormal evolution in a relationship, even if my character did not ease the process."\\

I nod, encouraging.\\

\speaker{Ambrosia} "You have all the keywords, the concepts. Not only people change with the passing of time, but their relationships also evolve. Ties which were inumaginable yesterday can grow freely today."\\

\speaker{Antiope} "Is this about your mentor?"\\

The question stuns me, but Antiope's mind is going downhill that way at full speed.\\

\speaker{Antiope} "I remember you evoking her a few times... And by the way you spoke about her, I got the idea you had a really close relationship. And now, we are at a festival where many musicians have gathered. You meet again did you?"\\

I have turned red as a tomato. I didnt' remember telling Antiope about Terpsichore... And especially not that much!\\

And now, she is looking at me with an unbearable knowing smile.\\

\speaker{Antiope} "Sorry, but I'm of no help for sentimental problems. But I'm sure you'll manage. Romance is your very life after all. Just be honest with her and everything will be alright."\\

No, no, no! You're completely misinterpreting!\\

Yet, I can't help by being troubled at how close of the truth she is somehow. Not so long ago, I did meet again Terpsichore by pure luck at a symposium where we both performed.\\

It was strange, awkward, nostalgic, interesting, passionate, enlightening. I'm absolutely certain I didn't tell a word of what happened that night to Antiope, but that fateful encounter helped me bring to closure some remaining issues about my past, and was indeed a major cobblestone in the road leading to this moment.\\

Antiope is now gently pushing me to go back to the party. She thankfully refrains from adding any "helpful" comment, but with each step leading me out of this sanctuary of tranquility, I feel sicker.\\

\begin{itemize}
	\item \glink{Forfeit}{Let go}
	\item \glink{Not so fast}{Fight back}
\end{itemize}

\gsection{Forfeit}

The perfect atmosphere, the perfect moment... And nothing went as expected.\\

I feel empty. Tired. Discouraged. I let myself by drawn back to the building, to the noise and the light, to the people and oblivion.\\

\glink{Masks}{Continue}

\gsection{Not so fast}

I take a good breath and stop her.\\

\speaker{Ambrosia} "No. That's not about that old affair. I'm talking about the future."\\

Antiope halts, surprised.\\

\speaker{Ambrosia} "I will not say that you're totally wrong about my apprenticeship..."\\

\speaker{Antiope} "Come on Ambrosia, I'm not the most subtle person ever, but I'm neither deaf or stupid."\\

\speaker{Ambrosia} "But that does not concern \emph{her}. That concerns \emph{you}."\\

Silence. Antiope stares at me, her gaze inscrutable. No escape now.\\

\speaker{Ambrosia} "Antiope, I love you."\\

\dcomment{
	Not so different from \glink{Declaration}{this end}. To be merged, with some subtle differences (Antiope didn't reveal anything about herself on this path, the deal is unfair with Ambrosia now transparent)
}

\chapter*{Endings}

\gsection{Masks}

\dcomment{
	Abrupt transition to the next morning
}

I feel groggy. I suppose I have slept. At least, I awoke on a straw mattress with no memory of a couple of hours. But I don't feel rested, not even close. With a lasting headache, I walk to the place Antiope and me agreed to meet once the party will be over.\\

And indeed, she is there. I'm relieved but also a little afraid of what will come next. And her reaction is the worst: Nothing. She does not act any different. She is exactly the same as she was two days prior.\\

Is she pretending or was everything what happened last night so unimportant to her? I can't say. And I don't want to know. She is there. We're talking like friends, still sharing the same intimacy. And I'm too much of a coward to risk breaking that again.\\

So I put on my own mask, I imbued myself of my own role, and I greet her cheerfully.\\

\dcomment{
	To rework, but I think there is something to do with the imagery of masks and theaters
}

\gsection{Friends}

A hard silence sets between us for a moment. We're both reflecting on our own mistakes I guess. Finally, I risk a few words.\\

\speaker{Ambrosia} "Is there anything else you need to tell me?"\\

She hesitates, then shakes her head.

\speaker{Antiope} "I have a lot more depressing stories in reserve, but that's far enough for one night. Ambrosia... Thanks. For listening to me. For not having freaked out when I acted like that."\\

\speaker{Ambrosia} "Don't worry. That's what friends are for."\\

\speaker{Antiope} "Still. Thanks."\\

Antiope smiles, and my heart races. I feel like a liar. I probably am, if only by omission. Yet, this may be for the best. She doesn't need a psychotic lover right now. Just an attentive, friendly ear. Or so do I try to convince myself.\\

The rest of the night is actually rather pleasant. We chitchat. I manage to bring her to one of the last musical representations. I myself sing. We go to bed relaxed and relieved.\\

On the next morning, we depart, as we planned to. Our adventure continues the same, except Antiope is a little more talkative and open. A good end probably.\\

And yet, dark at night, when alone with my thoughts, a sombre part of me still burns with jealousy, desire and regret, wondering what could have happened had I taken my chance at that time.\\

\dcomment{
	An acid pseudo good end does not sound bad, but I think there is still work to do to flesh it out properly
}

\gsection{Declaration}

I readjust my position, closing the distance between me and Antiope. Our faces are only separated by the width of a hand when I speak again.\\

\speaker{Ambrosia} "Antiope. You are not perfect. You have problems and you indeed made them worse by trying to solve them entirely by yourself. But you're not the only one. The world is full of idiots doing stupid things because they refuse to open their heart. Me included."\\

I swallow. Hard.\\

\speaker{Ambrosia} "The moment is not perfect, but it will never be. I cannot ask you to be honest with me when I'm not telling you the whole truth myself."\\

\speaker{Ambrosia} "Antiope."\\

\speaker{Ambrosia} "I love you."\\

Antiope's reaction is refreshingly simple. A flash of surprise succeeded by an extraordinary deluge of embarrassment. She turns almost comically pink from all the blood rising up to her head and mumbles some incomprehensible words. I smile warmly.\\

\speaker{Ambrosia} "Take your time. No need to hurry."\\

I feel calm. I have done it. The arrow has been shot, and there is nothing I can do to change its course now.\\

Antiope mumbles a little more, then breathes deeply and tries to compose an answer that makes some sense.\\

\speaker{Antiope} "I... First, thank you for telling me. I understand these are little words it takes a lot of courage to pronounce. Especially in our situation."\\

\speaker{Antiope} "I'm somehow flattered but... You're a extraordinary friend, a good \emph{travel} companion but... I have no romantic interest in you."\\

Silence.\\

\speaker{Antiope} "I'm sorry."\\

I try to keep my voice cool and articulate.\\

\speaker{Ambrosia} "I guessed so. Thank you for your frankness."\\

More awkward silence.\\

\speaker{Ambrosia} "If I may ask... Could you give me some time alone? I need it."\\

\speaker{Antiope} "Are you sure?"\\

\speaker{Ambrosia} "Certain. I will see you tomorrow, don't worry."\\

\speaker{Antiope} "Fine, I will go to sleep then. I've done enough idiocies for the night."\\

She stands up. Hesitates. Looks at me. I give her an encouraging nod. She utters some last civilities, and finally departs, not without looking back a few times, worry painted on her face. I disarm her with a smile each time.\\

Only when I'm certain she is far from here do I allow myself to break into tears.\\

\dcomment{
	Discrete transition
}

I'm late. By at least one hour. I already woke up late, but it's the time I spent cleaning my face which really put me in. The minutes flew by when I was erasing all traces of sadness, tiredness, despair from my face with a professional care, water, soap and an appropriate use of cosmetics. But now I look fresh and ready again.\\

Antiope is at the meeting point, and by the look of her own face, she did not take any action to wipe out the excesses of yesterday and has been worryingly waiting for quite a bit now. I greet her happily and she welcomes me with open arms and a touch of awkwardness.\\

As we walk together on the road, the distance between us becomes even more obvious. Or perhaps on the contrary it's the new proximity which is strange. Objectively, before, we never talked that much. I idealized her from far, and she never was really sociable.\\

Now, we're at least both conscious of the other existence as a real person. I will certainly never awake any passion in Antiope, and continuing to live by her side will certainly be a regular heart broker.\\

But still, at this moment, we need the other for practical if not sentimental reasons. So time to freeze my feelings, to lock them away and to drop the key.\\

After all, we can probably still be friends.\\

Perhaps.\\

\dcomment{
	To expand and rework. A conclusion must be a lot more memorable
}

\end{document}

