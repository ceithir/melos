\documentclass{report}

\usepackage[T1]{fontenc}
\usepackage[utf8]{inputenc}
\usepackage[greek,english]{babel}

\usepackage[hidelinks]{hyperref}
\hypersetup{
    colorlinks,
    linkcolor=blue,
}

\usepackage{comment}

\newcommand{\dcomment}[1]{
	\emph{#1}
	\\
}

\newcommand{\speaker}[1]{
	\textbf{#1}: 
}

\newcommand{\glink}[2]{
	\hyperref[#1]{#2}
}

\newcommand{\gsection}[1]{
	\section*{#1}
	\label{#1}
}

\title{LaTeX draft for \\"Ancient Greece Yuri Visual Novel"}
\author{Skarn}

\begin{document}

\selectlanguage{english}

\maketitle

\chapter{Where it all starts}

\dcomment{
	Night sky background
}

I don't know how and when it started.\\

\dcomment{
	Same background, one double paragraph of poetry filling most of the screen instead of scripts, with both the original Greek and its English translation, and the narration in the standard textbox
}

\begin{otherlanguage}{greek}
Ποικιλόθρον᾽ ἀθάνατ᾽ ᾽Αφρόδιτα,\\
παῖ Δίος, δολόπλοκε, λίσσομαί σε\\
μή μ᾽ ἄσαισι μήτ᾽ ὀνίαισι δάμνα,\\
πότνια, θῦμον.
\end{otherlanguage}

On the throne of many hues, Immortal Aphrodite,\\
child of Zeus, weaving wiles--I beg you\\
not to subdue my spirit, Queen,\\
with pain or sorrow
\\

I remember that morning on the road.

I had finally managed to have her talk a little about her past. For a second, for a few words, she opened the door of her heart to me, and I felt like I had unlocked the path to Elysium.\\

\begin{otherlanguage}{greek}
ἀλλὰ τυίδ᾽ ἔλθ᾽, αἴποτα κἀτέρωτα\\
τᾶς ἔμας αὔδως ἀίοισα πήλοι\\
ἔκλυες, πάτρος δὲ δόμον λίποισα\\
χρύσιον ἦλθες
\end{otherlanguage}

but come--if ever before\\
having heard my voice from far away\\
you listened, and leaving your father's\\
golden home you came
\\

I remember that afternoon in the pond.

We bathed together, not for the first time nor the last. But from that moment onwards, I was suddenly well aware of our common nudity and physical closeness.\\

\begin{otherlanguage}{greek}
ἄρμ᾽ ὐποσδεύξαια· κάλοι δέ σ᾽ ἆγον\\
ὤκεες στροῦθοι περὶ γᾶς μελαίνας\\
πύκνα δινεῦντες πτέρ᾽ ἀπ᾽ ὠράν᾽ αἴθε-\\
ρος διὰ μέσσω,
\end{otherlanguage}

in your chariot yoked with swift, lovely\\
sparrows bringing you over the dark earth\\
thick-feathered wings swirling down\\
from the sky through mid-air
\\

I remember that evening in a tavern overflowing with people.

Our table was crowd, but that did not bother her. Actually, she was smiling, joking, \emph{flirting} with the man next seat. I kept my best polite face. Inside my chest the infernal fire of jealousy and the chilling wind of despair were battling each other.\\

\begin{otherlanguage}{greek}
αἶψα δ᾽ ἐξίκοντο· σὺ δ᾽, ὦ μάκαιρα\\
μειδιάσαισ᾽ ἀθάνατῳ προσώπῳ,\\
ἤρε᾽ ὄττι δηὖτε πέπονθα κὤττι\\
δηὖτε κάλημι
\end{otherlanguage}

arriving quickly--you, Blessed One,\\
with a smile on your unaging face\\
asking again what have I suffered\\
and why am I calling again
\\

I remember that night in a particularly desolate region.

We had set up camp in a small grove. It was summer, it was hot, so we decide to sleep under the stars. She indeed did sleep. As for myself, I spent the entire night contemplating her, watching the slow up-and-down of her bust as she quietly breathes.\\

\begin{otherlanguage}{greek}
κὤττι μοι μάλιστα θέλω γένεσθαι\\
μαινόλᾳ θύμῳ· "τίνα δηὖτε †πείθω\\
ἄψ σ᾽ ἄγην† ἐς σὰν φιλότατα; τίς τ᾽, ὦ\\
Ψάπφ᾽, ἀδίκηει;
\end{otherlanguage}

and in my wild heart what did I most wish\\
to happen to me: "Again whom must I persuade\\
back into the harness of your love?\\
Sappho, who wrongs you?
\\

I remember our first encounter.

I knew nothing of her except of her skills with the spear and the snare. A shady character, refusing to reveal anything personal about herself. A probable fugitive on the run from the authorities, a potential thief, bandit, murderer.

Yet I still entrusted my safety to her. Out of necessity I tried to convince myself then. But perhaps, already, unconsciously, under the whispers of a part of myself older and far more powerful than my reasonable mind.\\

\begin{otherlanguage}{greek}
καὶ γάρ αἰ φεύγει, ταχέως διώξει,\\
αἰ δὲ δῶρα μὴ δέκετ᾽, ἀλλὰ δώσει,\\
αἰ δὲ μὴ φίλει, ταχέως φιλήσει,\\
κωὐκ ἐθέλοισα".
\end{otherlanguage}

For if she flees, soon she'll pursue,\\
she doesn't accept gifts, but she'll give,\\
if not now loving, soon she'll love\\
even against her will."
\\

I remember our last discussion.

We were waiting outside the stadium, before dawn, before this long day of festivities starts. More daring than usual, I had not so subtly directed the conversation on the various interpretations of my repertoire's works, and was watching her closely, attentive to any sign, any reaction which could have given me hope.

I saw none.\\

\begin{otherlanguage}{greek}
ἔλθε μοι καὶ νῦν, χαλέπαν δὲ λῦσον\\
ἐκ μερίμναν, ὄσσα δέ μοι τέλεσσαι\\
θῦμος ἰμέρρει, τέλεσον· σὐ δ᾽ αὔτα\\
σύμμαχος ἔσσο.
\end{otherlanguage}

Come to me now again, release me from\\
this pain, everything my spirit longs\\
to have fulfilled, fulfill, and you\\
be my ally
\\

I remember these moments, and many others, small and big, casual and special, every precious little brick of the time I spent with her.\\

\dcomment{
	Poem box disappears, leaving only the background and the standard textbox
}

A final strum, and the last note escapes the lyre, rises into the air, rings through the night. In its mourning, it echoes the end of the song, and is answered with applauses and cheers.\\

\dcomment{
	Background finally fade in
}

About twenty or thirty persons noisily show their appreciation of my recital of Sappho's classic. I have had bigger audiences, but at that time of the night, with everyone spread out in the many rooms and gardens of the building, that's not so bad.\\

I smile and salute, then climb down the rostrum as another artist ascends. In a couple of seconds, I cease to be the center of attention, fall back to the condition of simple spectator of someone else's show. I sit quietly on the side of a pedestal, half hidden in the shadow of the statue it bears.\\

And I almost break into a million pieces.\\

The \emph{Hymn to Aphrodite} is a powerful piece, one people expect and enjoy, one I have played many times in the past, and that I will probably play many times in the future too.

But for several months now, I have been unable to hum its melody without picturing a specific face, without imagining a certain pair of beautiful eyes looking at me while I'm singing, without dreaming about one unique person listening to the words between the lines and understanding the special meaning this song has for me.\\

And this is but the most harmless of my recent quirks. I have turned obsessive, possessive. I sleep little and badly, cannot bear the shortest of separations, constantly observing her and overinterpreting each and every of her words and actions.

I do my best to hide it, and as a professional artist trained to keep face and smile in all circumstances, I'm pretty sure I have managed to keep the mask of sanity and happiness on until now.\\

But deep inside, I'm crumbling, I'm on the verge of breaking, at the gates of madness. The flames are in me, they want to go out, but I keep them locked inside, and so they consume my soul.\\

More than the passion itself, it's the silence which is killing me. I haven't told anyone. Not her. Not anyone else, to be sure this would not reach her ears somehow. In fear of the consequences of such reveal.\\

I have observed her for days and nights. Yet, I never found a single hint that she may feel about me the same way that I feel about her.

But I have grasped enough to understand she was born in a culture of power and steel which does not value highly romance, a world where reproduction takes precedence over proper love. And that she still lives by the spirit of her education if not by its customs.\\

And I'm afraid. Very afraid. Of what will happen when she will discover my passion, my desires.\\

Afraid of bewilderment.

Afraid of rejection.

Afraid of fear.

Afraid of anger.\\

Afraid this will break our special relationship.\\

Afraid that by wanting too much of her, I will lost her whole.\\

\dcomment{
	Music change? Simply a lyre (harp) tuning sound effect as transition?
}

As a background noise for my worries, the man on the scene started playing. I recognize a composition from Alcaeus. A supposedly appropriate answer to my own choice, as he was Sappho's senior and rival. But a poor idea in the end, as the stanzas I hear simply cannot compare to the majesty of Sappho's texts. She was the uncontested Tenth Muse, and the artistic world still mourns her death a couple of decades ago.\\

The nagging melody nonetheless calms me down a little. I review my options. This afternoon, I took part in one of the main events of the festival, in the city theater, under the gazes of hundreds of people. A grand contest of poetry in whom many artists tried their luck, including the one distracting us right now.\\

Triumphing in such a competition is usually a life's goal for a singer, but this was not my main concern this year. However, I did plan my day with an extreme precision. As soon as my early and obligatory performance ended, I retreated back to the shadows, escaped discreetly the theater to hide myself in the nearby woods.\\

Then I sat, on the ground of a sunny clearing. Alone, while the whole city was directly or indirectly focused on the festival, from the shouting people on the theater's tiers to the humblest food vendor. To think. Think far away from the constant noise and movement of the ordinary life. Think without the weight of exhaustion and the terrors of the night overtaking me.\\

And after a few hours, when I knew I couldn't hide or flee anymore, I made my decision. It's this night or never. The moment couldn't be more perfect. Today, the city is – we will are – giving thanks to Aphrodite. Today, we both shone in our own specialty, prove ourselves to the world, reminded it we exist by our own rights. This night, right now, her mood should be at its best, on the edge of ecstasy.\\

If I don't talk to her this night, I know I will never have enough will to do it again. That my mouth will stay shut and my heart closed.\\

So... I just need to stand up, walk away from this quiet corner of the property where a few artists and enthusiasts gathered for some impromptu rehearsals, find her in the crowd, take her apart, and open my heart to her.\\

As simple as that.

\begin{itemize}
	\item \glink{Wait 1}{Wait}
	\item \glink{Go 1}{Go}
\end{itemize}

\gsection{Go 1}

I raise my head. I look at the nearby exit. So close. I just have to get up. Walk a few steps. Simple. Easy.

\begin{itemize}
	\item \glink{Wait 1}{Wait}
	\item \glink{Go 2}{Go}
\end{itemize}

\gsection{Go 2}

As I'm about to stand up, the artist on the scene rotates once more, and I hear the first note of a song I really like. Couldn't I just stay there at least until it ends?

\begin{itemize}
	\item \glink{Wait 1}{Listen}
	\item \glink{Go 3}{Go!}
\end{itemize}

\gsection{Go 3}

I try to stand up. My knees are shaking like a set of knucklebones. I'm shivering like an old sick lady. There is a jar of wine and a cup not far. Perhaps a drop could help.

\begin{itemize}
	\item \glink{Wait 1}{Have a drink}
	\item \glink{Go 3}{Just go!}
\end{itemize}

\gsection{Go 4}

I lash myself mentally. I'm still trying to push back the ineluctable. To flee what is hard and fearsome. But I cannot.\\

I gather my will.

I clench my teeth.

I rise.\\

I'm not stable on my feet but I nonetheless stumble on the short distance separating me from the main building. As soon as I've passed the separation door, that I'm plunged once again into the living heart of the party, noisy, hot, full of people, my faculties return to me. I'm no less afraid.

But once you jumped into the water, you have no more choice than to swim.\\

\glink{Searching for her}{Continue}\\

\gsection{Wait 1}

TODO


\begin{comment}
Most must be already quite drunk, especially the athletes who already enjoyed a \emph{generous} meal after their morning achievements.\\
 
I feel fire rising to my cheeks.

As per tradition, the morning was filled with numerous sportive events. Nothing as glorious as the Olympic games, but still a good deal of different competitions: jumping, wrestling, javelin throws, horse races... But I have little memories of all of them, completely eclipsed by the one which dried out my throat. The foot race.\\

More exactly, it's not the competition itself which left me unable to speak. I cannot even remember the name of the one who won it. But I do remember that she took part in it.

It's not that unusual for a woman to participate in such an event. Actually, there is often a token Spartan female athlete alongside the males, and they generally do quite well.

But that was not any woman. That was her.\\

These events are before everything else a present to the gods, and the competitors must be as resplendent as they are skilled, the sight of their nude bodies as much an offering as their actual performances.\\

And resplendent she was.\\

I helped her oiled herself before dawn, but there is a total difference between watching a greasy skin from close under the weak light of a candle and contemplating it shining under the bright sun.

A thousand reflections were stressing each line, each curve of her body, every movement of her sculpted muscles revealing a new feature, a new detail, a new wonder, in a never-ceasing display of beauty.
She had gathered her hair in a strict bun, but that only managed to stress out her face, her eyes, nose, lips, making them cuter than ever.\\

\dcomment{
	Description willingly vague while awaiting for illustrations.
	Of course, it's/it will be Ambrosia magnified vision, not necessarily the whole truth.
}

Enthralled, I was unable to look at anything else for the entire duration of the race, from the moment she entered the field to the last second before she retreated in the temple, out of sight.

Her expression when she covered the final meters, her reddish face transfigured from the intense action, her determined eyes looking far beyond the final line, is engraved in my memory. From that image alone, I can almost feel the shortness of her warmth breath, the drops of sweat running along her forehead, her brows, her cheeks, her lips...\\

I break the daydreaming at that point, remembering where I am, trying to appear attentive to the music.

But a little more self-control only saves face. It does not solve the real problem, in the same way this morning show did not start the fire in my heart.

It only poked a blaze which has been existing for months.\\

And, in spite of my mild efforts, my thoughts go wild once again.\\

I met her for the first time about four months ago. The group of wandering musicians I was with before had decided to remain in the Pieria region, renowned for its artists, but I was not so eager with the idea of setting up so young, and decided to continue my travels without them.\\

However, the roads of the peninsula are far from safe. The law pretty much ceases to exist as soon as the walls of the city are behind you and the wilderness is home to many four-legged and two-legged beasts.\\

Traveling alone never was an option, especially for someone like me, never trained in fighting and survival.

So I searched for a companion. And I found one quite easily.\\

She proposed herself as a mercenary, a bodyguard, even a porter, any job actually, as long as it would allow her to leave the area immediately.

She was clearly fleeing something, but she refused to talk about that. The only real information I got about her at that time was her name.\\

\foreignlanguage{greek}{Ἀντιόπη}. Antiope.\\

The safe option would have been to disregard such a shady offer. But she was a trained warrior, from one of those bellicose states where women are still taught the art of war alongside the men, and filled all my criteria for the ideal travel companion. She had both the skills I searched for and the will to roam the land. For free.\\

Maybe cold logic was not the only reason I accepted. Maybe, even back then, she was already bypassing my conscious mind to reach for a deeper and powerful part of myself. I cannot really say, and it's of little importance.\\

We departed together the next day. After a few hours of walking, I was in the middle of nowhere with an armed stranger who could have killed and looted me with the utmost ease.\\

She did not. She was as trustworthy that I thought – felt – she was.\\

After that, we traveled, ate, lived together for weeks and weeks, from city to city.

While in town, I was earning money with my songs and lyre. While in the wild, she was taking care of finding food and shelter.

A profitable association for both of us.\\

Yet it has slowly become unbearable for me. Not because I have grown to hate her manners and habits after watching them from close for months. The exact opposite actually.\\

At this very moment, I am completely enamored, infatuated, entranced, fascinated, captivated, ensorcelled, obsessed with her.\\

As soon as my thoughts are free from any immediate preoccupation, they fly right back at her. To be separated from her, even for a few hours, breaks my heart. Even in my sleep, I literally dream about her.\\

\end{comment}

\chapter{A talk}

\gsection{Searching for her}

I advance quickly through the many rooms of the building, getting around and stepping over the many revelers crossing my path with a consummate skill. I have worked at many similar parties before, and experience taught me how to move with ease despite the chaotic crowd.\\

I make a breach through human walls with a smile and a gentle but firm arm, adroitly avoid stepping on discarded leftovers and other unsavory things, nicely dismiss people asking for a song.\\

While my reflexes alone handle my movements and keep my pace, I gaze at every single person, my sight jumping from face to face, hunting the only prey worth it. A hundred different eyes meet mine, but not the ones I desire. Until they finally lock into perfection.\\

They may right now display the slight luster symptomatic of an important consumption of alcohol. But they still shine like twin suns. The reflected flames of the candles dance upon the dark and brown ocean, shifting with the emotions of their owner. Enlarged from the surprise of the sudden encounter. Contracted as she observes the intruder. Sparkling when she recognizes me.\\

She instantly drops off her plate of food, rises from her makeshift seat, and jumps to hug me. Her warm arms vigorously lock behind my back in a frank embrace, bringing my face mere centimeters of her own.\\

\dcomment{
	First actual appearance of Antiope, with a simple zoom for the close-up
}

This is a purely friendly greeting, yet it takes all my will to keep my heartbeat in check. After an ephemeral eternity, she releases me and speaks.\\

\speaker{Antiope} "Ambrosia! Where have you been? You just disappeared after your performance at the poetry contest. I was unable to find you anywhere. That's so unlike you!"\\

I savor my own name in her mouth like it was the divine nectar it refers to. I feel guilty for it, but I enjoy far too much the fact that she did remark that I was not there anymore, and that she cared about it.\\

I smile timidly. I escaped for her, to have some time alone to think about the future of both of us at peace. But is that something I should really say so bluntly.

\begin{itemize}
	\item \glink{Approaching the rift}{Say that I needed the solitude}
	\item \glink{Sport}{Avoid the subject and praise her for her performance this morning}
\end{itemize}

\gsection{Approaching the rift}

\speaker{Ambrosia} "I needed a break. To put some distance between me and the crowd. To think."\\

I barely recognize my own voice. It sounds so... depressed. She blinks, grabs my hand, her face suddenly serious, concerned.\\

\speaker{Antiope} "Ambrosia, are you alright? Is there something wrong? You're not acting like yourself. Are you sick? Has someone done something to you?"\\

Yes dummy, someone has. That someone is you.

\begin{itemize}
	\item \glink{It's nothing}{Reassure her}
	\item \glink{The garden}{Ask if we can discuss somewhere else}
\end{itemize}

\gsection{Sport}

TODO

\gsection{It's nothing}

TODO

\gsection{The garden}

\speaker{Ambrosia} "Can we talk about this in a calmer and quieter place?"\\

This statement does not reassure her a bit. She looks around, appears to notice the many people around us, even if most actually pay no attention to our little show, busy that they are with their friends or drinks, but still contributing to the stifling atmosphere of the overfull room.\\

She then reacts with her natural straightforwardness. Grabbing me by the wrist, she walks right through the crowd, taking the shortest route to the nearest exit. If I got here like a breeze, slipping in cracks and leaving no trace of my passing behind me, Antiope is a ram, a spur, elbowing her way out, breaking any resistance like a charging bull.\\

I should probably scold her for her rude behavior. But I cannot resolve myself to do it. Partly because I realize I really want to go out. Partly because I'm somehow happy, even if this happiness is tainted with guilt, that she is worried about me, that did not hesitate for one second to discard all decorum for me.

TODO: Continue

\end{document}

