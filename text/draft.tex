\documentclass{report}

\usepackage[T1]{fontenc}
\usepackage[utf8]{inputenc}
\usepackage[greek,english]{babel}

\usepackage[hidelinks]{hyperref}
\hypersetup{
    colorlinks,
    linkcolor=blue,
}

\newcommand{\dcomment}[1]{
	\emph{#1}
	\\
}

\newcommand{\speaker}[1]{
	\textbf{#1}: 
}

\title{LaTeX draft for \\"Ancient Greece Yuri Visual Novel"}
\author{Skarn}

\begin{document}

\maketitle

\chapter*{Where it all starts}

\selectlanguage{english}

\dcomment{
	Text fills most of the screen, black background, one double paragraph by screen, with both the original Greek and its English translation
}

\begin{otherlanguage}{greek}
Ποικιλόθρον᾽ ἀθάνατ᾽ ᾽Αφρόδιτα,\\
παῖ Δίος, δολόπλοκε, λίσσομαί σε\\
μή μ᾽ ἄσαισι μήτ᾽ ὀνίαισι δάμνα,\\
πότνια, θῦμον.
\end{otherlanguage}

On the throne of many hues, Immortal Aphrodite,\\
child of Zeus, weaving wiles--I beg you\\
not to subdue my spirit, Queen,\\
with pain or sorrow
\\

\begin{otherlanguage}{greek}
ἀλλὰ τυίδ᾽ ἔλθ᾽, αἴποτα κἀτέρωτα\\
τᾶς ἔμας αὔδως ἀίοισα πήλοι\\
ἔκλυες, πάτρος δὲ δόμον λίποισα\\
χρύσιον ἦλθες
\end{otherlanguage}

but come--if ever before\\
having heard my voice from far away\\
you listened, and leaving your father's\\
golden home you came
\\

\begin{otherlanguage}{greek}
ἄρμ᾽ ὐποσδεύξαια· κάλοι δέ σ᾽ ἆγον\\
ὤκεες στροῦθοι περὶ γᾶς μελαίνας\\
πύκνα δινεῦντες πτέρ᾽ ἀπ᾽ ὠράν᾽ αἴθε-\\
ρος διὰ μέσσω,
\end{otherlanguage}

in your chariot yoked with swift, lovely\\
sparrows bringing you over the dark earth\\
thick-feathered wings swirling down\\
from the sky through mid-air
\\

\begin{otherlanguage}{greek}
αἶψα δ᾽ ἐξίκοντο· σὺ δ᾽, ὦ μάκαιρα\\
μειδιάσαισ᾽ ἀθάνατῳ προσώπῳ,\\
ἤρε᾽ ὄττι δηὖτε πέπονθα κὤττι\\
δηὖτε κάλημι
\end{otherlanguage}

arriving quickly--you, Blessed One,\\
with a smile on your unaging face\\
asking again what have I suffered\\
and why am I calling again
\\

\begin{otherlanguage}{greek}
κὤττι μοι μάλιστα θέλω γένεσθαι\\
μαινόλᾳ θύμῳ· "τίνα δηὖτε †πείθω\\
ἄψ σ᾽ ἄγην† ἐς σὰν φιλότατα; τίς τ᾽, ὦ\\
Ψάπφ᾽, ἀδίκηει;
\end{otherlanguage}

and in my wild heart what did I most wish\\
to happen to me: "Again whom must I persuade\\
back into the harness of your love?\\
Sappho, who wrongs you?
\\

\begin{otherlanguage}{greek}
καὶ γάρ αἰ φεύγει, ταχέως διώξει,\\
αἰ δὲ δῶρα μὴ δέκετ᾽, ἀλλὰ δώσει,\\
αἰ δὲ μὴ φίλει, ταχέως φιλήσει,\\
κωὐκ ἐθέλοισα".
\end{otherlanguage}

For if she flees, soon she'll pursue,\\
she doesn't accept gifts, but she'll give,\\
if not now loving, soon she'll love\\
even against her will."
\\

\begin{otherlanguage}{greek}
ἔλθε μοι καὶ νῦν, χαλέπαν δὲ λῦσον\\
ἐκ μερίμναν, ὄσσα δέ μοι τέλεσσαι\\
θῦμος ἰμέρρει, τέλεσον· σὐ δ᾽ αὔτα\\
σύμμαχος ἔσσο.
\end{otherlanguage}

Come to me now again, release me from\\
this pain, everything my spirit longs\\
to have fulfilled, fulfill, and you\\
be my ally
\\

\dcomment{
	Back to classic presentation, with textbox at the bottom, background stays black
}

A final strum, and the last note escapes the lyre, rises into the air, rings through the night.

In its mourning, it echoes the end of the song, and is answered with applauses and cheers.\\

\dcomment{
	Background finally fade in
}

About twenty or thirty persons noisily show their appreciation of my recital of Sappho's classic. I have had bigger crowds, but at that time of the night, with everyone spread out in the many rooms and gardens of the building, that's not so bad.

Smiling, I climb down the pedestal while another artist ascends.\\

Like me, he took place in the main event several hours ago, a contest of skills between many poets and poetesses, on the stage of the city theater, under the gazes of the gods and of hundreds of men and women.

But now we are all simply playing for fun, with only the starry sky and a few enthusiasts as audience and jury.\\

The man chose a composition form Alcaeus, an appropriate continuation for my own song, as he was Sappho's senior/friend/rival.

The exact details of their relationship have been the subject of many rumors, as every aspect of the life of the life of the Tenth Muse did.

So much actually that, while she only passed away a couple of decades ago, she is already shrouded in myth, a new Homer from a distant island on the other side of the sea.\\

However, while a part of my mind is delving into these educated thoughts and tries to listen to the performance with the due respect it deserves, most of me does not really care.

My heart still beating faster from the stress and excitation that every public show never miss to bring, I keep scanning the crowd frantically, searching for her.\\

The \emph{Hymn to Aphrodite} is a powerful piece, one people expect and enjoy, one I have played many times in the past, and that I will probably play many times in the future too.

But for several months now, I have been unable to hum its verse, its melody, without picturing a specific face, without imagining a certain pair of beautiful eyes looking at me while I'm singing, without dreaming about one unique person listening to the words between the lines and understanding the special meaning this song has for me.

Only when under the fiery fury of an actual performance, when all my focus is put on the technique, these thoughts recess for a time. But they come back as soon as it ends, more powerful than before.\\

...
\\

Of course, she is not there. Only those most passionate about poetry have gathered to this calm alcove. The others are partying all over the place, eating, gambling, joking, while their cups of wine are constantly refilled.

Most must be already quite drunk, especially the athletes who already enjoyed a \emph{generous} meal after their morning achievements.\\
 
I feel fire rising to my cheeks.

As per tradition, the morning was filled with numerous sportive events. Nothing as glorious as the Olympic games, but still a good deal of different competitions: jumping, wrestling, javelin throws, horse races... But I have little memories of all of them, completely eclipsed by the one which dried out my throat. The foot race.\\

More exactly, it's not the competition itself which left me unable to speak. I cannot even remember the name of the one who won it. But I do remember that she took part in it.

It's not that unusual for a woman to participate in such an event. Actually, there is often a token Spartan female athlete alongside the males, and they generally do quite well.

But that was not any woman. That was her.\\

These events are before everything else a present to the gods, and the competitors must be as resplendent as they are skilled, the sight of their nude bodies as much an offering as their actual performances.\\

And resplendent she was.\\

I helped her oiled herself before dawn, but there is a total difference between watching a greasy skin from close under the weak light of a candle and contemplating it shining under the bright sun.

A thousand reflections were stressing each line, each curve of her body, every movement of her sculpted muscles revealing a new feature, a new detail, a new wonder, in a never-ceasing display of beauty.
She had gathered her hair in a strict bun, but that only managed to stress out her face, her eyes, nose, lips, making them cuter than ever.\\

\dcomment{
	Description willingly vague while awaiting for illustrations.
	Of course, it's/it will be Ambrosia magnified vision, not necessarily the whole truth.
}

Enthralled, I was unable to look at anything else for the entire duration of the race, from the moment she entered the field to the last second before she retreated in the temple, out of sight.

Her expression when she covered the final meters, her reddish face transfigured from the intense action, her determined eyes looking far beyond the final line, is engraved in my memory. From that image alone, I can almost feel the shortness of her warmth breath, the drops of sweat running along her forehead, her brows, her cheeks, her lips...\\

I break the daydreaming at that point, remembering where I am, trying to appear attentive to the music.

But a little more self-control only saves face. It does not solve the real problem, in the same way this morning show did not start the fire in my heart.

It only poked a blaze which has been existing for months.\\

And, in spite of my mild efforts, my thoughts go wild once again.\\

I met her for the first time about four months ago. The group of wandering musicians I was with before had decided to remain in the Pieria region, renowned for its artists, but I was not so eager with the idea of setting up so young, and decided to continue my travels without them.\\

However, the roads of the peninsula are far from safe. The law pretty much ceases to exist as soon as the walls of the city are behind you and the wilderness is home to many four-legged and two-legged beasts.\\

Traveling alone never was an option, especially for someone like me, never trained in fighting and survival.

So I searched for a companion. And I found one quite easily.\\

She proposed herself as a mercenary, a bodyguard, even a porter, any job actually, as long as it would allow her to leave the area immediately.

She was clearly fleeing something, but she refused to talk about that. The only real information I got about her at that time was her name.\\

\foreignlanguage{greek}{Ἀντιόπη}. Antiope.\\

The safe option would have been to disregard such a shady offer. But she was a trained warrior, from one of those bellicose states where women are still taught the art of war alongside the men, and filled all my criteria for the ideal travel companion. She had both the skills I searched for and the will to roam the land. For free.\\

Maybe cold logic was not the only reason I accepted. Maybe, even back then, she was already bypassing my conscious mind to reach for a deeper and powerful part of myself. I cannot really say, and it's of little importance.\\

We departed together the next day. After a few hours of walking, I was in the middle of nowhere with an armed stranger who could have killed and looted me with the utmost ease.\\

She did not. She was as trustworthy that I thought – felt – she was.\\

After that, we traveled, ate, lived together for weeks and weeks, from city to city.

While in town, I was earning money with my songs and lyre. While in the wild, she was taking care of finding food and shelter.

A profitable association for both of us.\\

Yet it has slowly become unbearable for me. Not because I have grown to hate her manners and habits after watching them from close for months. The exact opposite actually.\\

At this very moment, I am completely enamored, infatuated, entranced, fascinated, captivated, ensorcelled, obsessed with her.\\

As soon as my thoughts are free from any immediate preoccupation, they fly right back at her. To be separated from her, even for a few hours, breaks my heart. Even in my sleep, I literally dream about her.\\

I don't know how and when it started. I don't even know when exactly I realized I had fallen in love.\\

I remember that morning on the road.

I had finally managed to have her talk a little about her past. For a second, for a few words, she opened the door of her heart to me, and I felt like I had unlocked the path to Elysium.\\

I remember that afternoon in the pond.

We bathed together, not for the first time nor the last. But from that moment and onwards, I was suddenly well aware of our common nudity and physical closeness.\\

I remember that night in a particularly desolate region.

We had set up camp in a small grove. It was summer, it was hot, so we decide to sleep under the stars. She indeed did sleep. As for myself, I spent the entire night contemplating her, watching the slow up and down of her bust as she quietly breathes like some hypnotic pendulum.\\

I remember that evening in a tavern overflowing with people.

Our table was crowd too, but that did not bother her. Actually, she was smiling, joking, \emph{flirting} with the man next seat. I kept my best polite face. Inside my chest the infernal fire of jealousy and the chilling wind of despair were battling each other.\\

I remember many other places and times, small and big fragments of passionate memories, filling bit by bit the caverns of my mind, clogging everything else.\\

And I'm on the verge of breaking, at the gates of madness. The flames are in me, they want to go out, but I keep them locked inside, and so they consume my soul.\\

I haven't told anyone. Not her, nor anyone else, in the fear she would hear of them in the end.\\

I have observed Antiope for days and nights. Yet, I never found a single hint that she may feel about me the same way I feel about her.

But I have grasped enough to understand she was born in a culture of power and steel which does not value highly romance, a world where reproduction takes precedence over proper love.\\

And I'm afraid. Very afraid. Of what will happen when she will discover my passion, my desires.\\

Afraid of bewilderment.

Afraid of rejection.

Afraid of fear.

Afraid of anger.\\

Afraid this will break our special relationship.\\

Afraid that by wanting too much of her, I will lost her whole.\\

\chapter*{A talk}

\section*{Alone in the crowd}

I advance quickly through the many rooms of the building, getting around and stepping over the many revelers crossing my path with a consummate skill. I have worked at many similar parties before, and experience taught me how to move with ease despite the chaotic crowd.\\

I make a breach through human walls with a smile and a gentle but firm arm, adroitly avoid stepping on discarded leftovers and other unsavory things, nicely dismiss people asking for a song.\\

While my reflexes alone handle my movements and keep my pace, I gaze at every single person, my sight jumping from face to face, hunting the only prey worth it. A hundred different eyes meet mine, but not the ones I desire. Until they finally lock into perfection.\\

They may right now display the slight luster symptomatic of an important consumption of alcohol. But they still shine like twin suns. The reflected flames of the candles dance upon the dark and brown ocean, shifting with the emotions of their owner. Enlarged from the surprise of the sudden encounter. Contracted as she observes the intruder. Sparkling when she recognizes me.\\

She instantly drops off her plate of food, rises from her makeshift seat, and jumps to hug me. Her warm arms vigorously lock behind my back in a frank embrace, bringing my face mere centimeters of her own.\\

\dcomment{
	A simple zoom in on the sprite should do the job. Cheap, but efficient.
}

This is a purely friendly greeting, yet it takes all my will to keep my heartbeat in check. After an ephemeral eternity, Antiope releases me and speaks.\\

\speaker{Antiope} "Ambrosia! Where have you been? You just disappeared after your performance at the poetry contest. I was unable to find you anywhere. That's so unlike you!"\\

I smile timidly. It's true that I sneaked out as soon as my obligations were fulfilled, walking away for a long time, only returning at dusk. But I did so to have some time alone, to calm down myself after the frenzy of the morning, to think about the future of both of us at peace.

\begin{itemize}
	\item \hyperref[A1]{Reveal that I needed the solitude}
	\item \hyperref[A2]{Avoid the subject and praise her for her own achievement}
\end{itemize}

\section*{We need to talk}

\label{A1}

\speaker{Ambrosia} "I needed a break. To put some distance between me and the crowd. To think."\\

I barely recognize my own voice. It sounds so... depressed. Antiope blinks, and grabs my hand, her face suddenly serious, concerned.\\

\speaker{Antiope} "Ambrosia, are you alright? Is there something wrong? You're not acting like yourself. Are you sick? Had someone done something to you?"\\

Yes dummy, someone has. That someone is you.

\begin{itemize}
	\item \hyperref[A3]{Reassure her}
	\item \hyperref[A4]{Ask if we can discuss somewhere else}
\end{itemize}

\section*{Praise the beauty of the body}

\label{A2}

\section*{Everything's alright}

\label{A3}

\section*{The garden}

\label{A4}

\speaker{Ambrosia} "Can we talk about this in a calmer and quieter place?"\\

This statement does not reassure Antiope a bit. She looks around, appears to notice the many people around us, even if most actually pay no attention to our little show, busy that they are with their friends or drinks, but still contributing to the stifling atmosphere of the overfull room.\\

She then reacts with her natural straightforwardness. Grabbing you by the wrist, she walks right through the crowd, taking the shortest route to the nearest exit. If you got here like a breeze, slipping in cracks and leaving no trace of your passing behind you, Antiope is a ram, a spur, elbowing her way out, breaking any resistance like a charging bull.\\

I should probably scold her for her rude behavior. But I cannot resolve myself to do it. Partly because I realize I really want to go out. Partly because I'm somehow happy, even if this happiness is tainted with guilt, that she is worried about me, that did not hesitate for one second to discard all decorum for me.

\end{document}

